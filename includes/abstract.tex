\begin{abstract}
In the recent years, awareness of the issue of Environmental Pollution has increased, and research shows that not enough has been done, until now, to reduce pollution. In this domain, the monitoring of air quality is fundamental to provide data which can be used to most effectively guide our efforts to reduce air pollution. At this time the monitoring of air quality is usually performed via stationary ground-mounted air pollution stations. However, research(??) has shown that air pollution can vary greatly at different heights, for this reason the \gls{aria} project is aiming to develop a system to measure vertical gradients of air pollutants using vertical swarms of drones. The \gls{aria} project solution is a low-cost monitoring system based on \gls{cots} sensors and on multiple cheap drone platforms. The system is equipped with $PM_2.5$ and $PM_10$ sensors to monitor the particulate concentration and several other gas sensors (such as $NO$, $NO_2$, $CO$, etc.) and the use of \gls{uavs} allows to build a 3D map of pollutants in a specific area. This could prove very useful around buildings in urban areas and possibile polluting plants in industrial areas. In this thesis are presented the system platform, the software implementation and a test flight.
\end{abstract}