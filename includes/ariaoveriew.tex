\chapter{ARIA project}
\section{Overview}
ARIA project was created by a group of students from the Department of Industrial Engineering, University of Padova, under the suggestion and guidance of personnel staff of the Center for Space Studies and Activities (CISAS) of the same University. The core motivation that brought together these students was the desire of researching new fields of application for drone technology.ARIA project’s scenario is investigation about drones usage within air quality monitoring. Environmental pollution is becoming every day more threatening for our health and we wanted to develop a tool to monitor it in 3D. the project is funded by the Department of Industrial Engineering of our University.
\section{Proposal presentation}

As for system which use only \gls{uavs}, however, a single UAV, is very limited in its performance due to its coverage, energy autonomy and small selection of sensors.  Swarms, instead, can provide full coverage of an area, while coordinating the best routes to visit each sensing node. Equipping different drones with different sensors is far easier and more flexible than having one doing everything on its own. The differentiating factor of the \gls{aria} project solution is the use of vertical swarms to monitor pollution at different heights, which is not a popular topic.