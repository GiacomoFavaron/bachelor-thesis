\chapter{ARIA project}
\section{Overview}
ARIA project was created by a group of students from the Department of Industrial Engineering, University of Padova, under the suggestion and guidance of personnel staff of the Center for Space Studies and Activities (CISAS) of the same University. The core motivation that brought together these students was the desire of researching new fields of application for drone technology. ARIA project’s scenario is investigation about drones usage within air quality monitoring. Environmental pollution is becoming every day more threatening for our health and we wanted to develop a tool to monitor it in 3D. the project is funded by the Department of Industrial Engineering of our University.
\section{Proposal presentation}
ARIA project’s ultimate goal is gathering data about the
values of major pollutants in urban areas at various heights.
Our tool will be a vertical drone swarm equipped with low-cost sensors and deployable by utilizing GPS coordinates. The
information will be stored in a database, so that they can be
post-processed and published whenever appropriate. The most
interesting activity could be creating a contemporary vertical
profile of pollutants concentrations for various time-periods.
We’ve chosen to work with UAVs because after reading
literature about environmental monitoring, we found omissions
in Wireless Sensor Networks. The use of this new technology
was already recommended thanks to its speed, mobility and
capability to fly at different heights. The main constraint was
the unavailability of low-cost and low bulk drones.
To differentiate ourselves from previous studies (since our
knowledge would not be able to compete with them anyway)
we wanted to explore vertical swarms. The study of pollution
at different heights is not a popular topic and we wanted to
investigate it.
The task for the multicopter will have a standard implementation: usually it will be a simple request to move towards some
GPS coordinates, hover there while collecting samples and then
return to the base whenever the time is up or the battery is too
low, or move through different GPS waypoints, hovering at each one to allow the sensors to adjust and collect the data. 
The gas sensors have a response time of around 30s which needs to be taken into consideration.
The flight will be planned before departure from a
terminal. The vertical deployment doesn’t need to be extremely
precise, since sensor’s accuracy is not sufficient for slight
misplacements to matter, and  . Since our approach will be careful
and gradual, each entity belonging to the swarm will not
communicate with the others. Considering this fact, we know
the term ”swarm” is being used inappropriately. Wireless
communication and real swarm implementation will follow as
the project unfolds.

