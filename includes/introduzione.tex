\chapter{Introduction}
% TODO:
% - Trovare ref punti di domanda
Air pollution is caused by different typologies of gas pollutants that are present in the first meters (<150 m) of the atmosphere and cause therefore damages to humans and environment. As air pollution is becoming the largest environmental health risk, the monitoring of air quality has drawn much attention in both laboratory studies and specific field tests and data collection campaigns. Government agencies and local administrations have, generally, provided and used monitoring stations on dedicated sites in cities and urban areas. Usually the studies have been conducted using fixed stations that are very reliable but produce only coarse-grained 2D monitoring, with several kilometers between two monitoring stations; or the stations monitor the same local area for long periods.
Other approaches show that applications using simple system of sensors have been developed to monitor the fine-grained air quality using densely deployed sensors [? ], [? ]. In any case, the fixed sensor station may achieve high precision, but have high cost and require maintenance and suffer especially for lack of mobility.
Furthermore, these approaches don't account for the vertical gradients of air pollution levels. As shown in research (?, ?) the concentrations of air pollutants can vary greatly at different heights and this is a sensitive factor in circumstances such as buildings in urban areas and possible polluting plants in industrial areas.
The usage of Unmanned Aerial Vehicles (UAVs) has been particularly rich in the latest years due to their flexibility, mobility and affordable cost. Current monitoring systems are not able to satisfy every need of modern cities and industrial areas and UAVs are valuable supporting elements in this scenario.
In terms of urban conditions, which is the main subject of the present study, UAVs can be used to measure environmental parameters such as illumination, wind speed, temperature, humidity, air quality [? ] and much more. In any case, for a complete analysis, both ground sensing and aerial sensing are necessary to provide 3D mapping and gas profiling. In our ARIA project, we equipped with the same set of sensors the devices that execute sensing on the ground, and the systems that execute aerial sensing on board the \gls{uavs}.
%, which we are exploring to deploy in vertical swarms, to measure pollution levels at different heights.
The fixed ground sensing suite is able to collect data in a continuous way, but the air quality of the higher levels of air off the ground cannot be detected, so the contemporary use of drones is mandatory. Aerial sensing, on the contrary, is able to sense the air quality off the ground, but it cannot be executed for very long periods due to the high consumption of battery power and human time. By merging the potentialities of these two systems of sensing suites, a better set of data can be collected [? ]. A trade off on the possible sensors and UAVs has been performed and quadcopters are the preferred platform for monitoring because of their simplicity, low cost and hovering capabilities. On the contrary a possible bias of data is due to the the influence of air jets created by the rotor rotation or by the electromagnetic field generated by the antennas present on board. The problem of choosing the best location of the sensors is examined in [? ] based on the physical structure of the drones. Our approach is to use an extension on which we fix the sensors in order to suck the air away of the main air jets.

ARIA project was created by a group of students from the Department of Industrial Engineering, University of Padova, under the suggestion and guidance of personnel staff of the Center for Space Studies and Activities (CISAS) of the same University. The core motivation that brought together these students was the desire of researching new fields of application for drone technology. ARIA project has been carried on figuring this scenario: air quality monitoring.
\section{Related works and state of the art}
\subsection{Air Pollution}
Air pollution is extremely complex to evaluate and there are many polluting substances in the atmosphere. The \gls{epa} (of United States) takes these 6 in consideration in its studies:
\begin{center}
\begin{tabular}{ |c|c|c| }
    \hline
    Chemical symbol & Substance & Characteristics \\ [0.5ex]
    \hline
    CO & Carobon Monoxide & Colorless, odorless gas \\
    $NO_2$ & Nitrogen Dioxide & Highly reactive gas \\
    $O_3$ & Ozone & Pale blue gas \\
    $SO_2$ & Sulfur Dioxide & Colorless, irritating smell gas \\
    $PM_2.5$ and $PM_10$ & Particulate Matter & Inhalable particles \\
    Pb & Lead & Metal particles \\
\end{tabular}
\end{center}
\subsection{Low-cost sensors}
The \gls{epa} also provides the Air Sensor Guidebook\cite{williams2014-air} which gives extensive information on air quality and low-cost sensors.

\cite{evangelatos2015airborne}, \cite{8675167}, \cite{8662050} propose different implementation of a \gls{wsn} using \gls{uavs}. The differentiating factor of the \gls{aria} project solution is the use of vertical swarms to monitor pollution at different heights, which is not a popular topic.
\section{Dissertation structure}
This dissertation describes the \gls{aria} project solution for the monitoring of air pollution. It is divided into 6 chapters:
\begin{itemize}
    \item Chapter 1 describes the introduction, an overview on the topic of air pollution, the motivation to approach the problem, the motivation of the proposed solution, related works, the state of the art and the dissertation objective.
    \item Chapter 2 describes the system architecture, that is the \gls{uavs} that are being used, their design, specifications and functionality.
    \item Chapter 3 describes the sensor payload, the motivation of the adopted sensors and their use.
    \item Chapter 4 describes the software implementation for the data collection of the sensors and the communication of the \gls{uavs}.
    \item Chapter 5 shows the results of a test flight using the proposed solution.
    \item Chapter 6 presents what conclusions can be taken after all the developed work, and what improvements can be done in the future.
\end{itemize}
\section{Dissertation objective}
The objective of this dissertation is to describe the solution proposed by the \gls{aria} project for air pollution monitoring, in particular the software impelementation, to show preliminary results and discuss their revelance in future applications.